\section{Conclusion}

Throughout this project the goal was to build a fast, lightweight robot that could compete well against other robots. Through the construction we managed to make the robot fast through the use of gears, and lightweight by carefully designing the chassis and constantly optimising its design. The strategy utilised this speed and agility well, especially with the use of the Potential Fields algorithm, which gave our robot a quick and smooth movement technique. \linebreak

One of the main issues we came across was working as a team, in the early stages finding it hard to find times to work together and deciding who was responsible for specific areas of development and decisions. Bringing in a team manager improved this a great deal, and by increasing our efforts of communicating via email and the website, alongside making the code more understandable, we were eventually able to co-operate as a team successfully. Good communication between team members developed during project, use of Extreme Programming methods (Pair Negotiation, Unit Tests, Pair programming, etc.) were the key factors in successful completion of our goals.\linebreak

 These communication skills are likely to be the main disposable skills that we will take away from this project, other skills including building robots, integrating different areas of code, and how to deal with programming and testing until the early hours of the morning!