\section{Group Organisation}

It was clear from the beginning that organisation and planning would be the most important factor in laying the groundwork for a successful team project. Establishing regular group meetings would be key in keeping track of our progress. Group structure, timetable planning and methods of communication were the main topics of conversation during our first group meeting. Understanding each others strengths and weaknesses would be crucial to establishing a productive team hierarchy. It was concluded that having three teams (Robot Construction, Vision, and Motion and Behaviour) would satisfy the skills available from the members of our group, and divide the large scope of the project into more easily manageable sub-tasks. \linebreak

After initial issues of team communications where members would often discuss areas, but no definite actions were taken, we realised that it would be beneficial to have an official Group Manager, in order to improve group dynamics and focus on the progress of the project. This was very useful as the manager was able to keep track of milestones and deadlines and ensure that team members kept to them, designating tasks to individual members to avoid confusion about who was doing which task.\\

With regards to communication between group members, our mentor prepared a mailing list for initial conversations to take place. This proved to be the main direct form of contact between team members (excluding talking in person of course!), and was used daily throughout the project. A private Facebook group was created for our team, such that simple messages such as notice alerts, or timetable plans, could be effectively transmitted to everyone due to Facebook being a medium which every member used regularly. A wiki-like website was set up using Google Sites to handle task updates, future ideas, and individual logs. The website was integral to maintaining group communication, with each individual member being encouraged to keep a detailed log of all their work, allowing everyone to be kept up to date on the team’s progress as a whole. These individual logs, as well as group logs for the three sub-groups, provided a detailed review of the status of the project, that any group member could use to easily check up on any work that had been done in their absence. This helped us ensure that all the subgroups were working in tandem, and greatly eased system integration. The complete transparency of the project allowed through the individual and team logs was very useful, meaning team members always had a good idea of where the project was at, and where it was going. \linebreak

An SVN repository was also founded, not only for code, but for reports, documentation and data from testing. We moved away from the university provided SVN to a professional company SVN, due to the unreliability of the uptime of the university SVN.\linebreak 

Finallly, and possibly most importantly, throughout the project the code was usefully commented, and team members strove to produce code which was easy for other members to understand. This meant any member would be able to look and the code, understand it, see where it could be improved, and add to to it. TODO's were often left in the code to simplify finding areas that needed adapting, and instructing other members about what needed to be added/changed in that section of the code.
