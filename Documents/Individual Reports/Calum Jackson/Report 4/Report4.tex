\documentclass[12pt]{article}
\usepackage[a4paper]{ geometry}
\geometry{top=0.0in, bottom=0.7in, left=0.7in, right=0.7in}
\fontsize{12}{12}
\begin{document}
\title{Individual Report 4}
\author{Calum Jackson (s0812597) - 
SDP Group 12}
\maketitle
\begin{flushleft}

\section{Introduction}
This report, the fourth in the series, will document the changes made to the robot since Milestone 3, changes to the strategy code, and the robots performance in the friendly match and Milestone 4. The team had two main objectives:
\begin{itemize}
\item Ensure the robot was capable of competing in the upcoming friendly matches.
\item Implement a intercepting strategy for milestone 4.
\end{itemize}

\section{Strategy}
Focusing on the friendly match, the strategy code was improved in numerous areas, including:
\begin{itemize}
\item Methods were created to overcome situations such as if a destination point was of the pitch, if the opponent has possession of the ball, or if the touch sensors were activated.
\item Strategy classes were integrated together using a MainStrategy class, which uses a state system to decide what strategy needs to be called to deal with the current on-pitch situation
\item Code was cleaned to improve usefulness of methods and remove useless code (such as using is
\item The method calculating the obstacle avoidance point were improved to implement a dynamic point instead of a static point. This was done by calculating the --
\end{itemize}

\section{Robot Changes}
Touch sensors were added at the front of the robot, and implemented in the code, thus preventing the robot continually crashing into an obstacle. During the friendly matches it was apparent the plate holder was impeding access to the NXT brick, wasting game time when the robots were being reset. This was corrected by putting the plate holder on a hinge, giving instant access. The kicker has again been altered 






\end{flushleft}
\end{document}